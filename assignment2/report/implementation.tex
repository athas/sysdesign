\documentclass[12pt]{article}
 
\author{
        Troels Henriksen, Troels Visti Thrane, Kasper Middelboe Petersen
}
\date{\today}

\usepackage{graphicx}
 
\title{Implementation}
 
\begin{document}
 
\maketitle

\section{General Summary}
Write me!

\section{Interesting aspects}

\subsection{add\_version}

This functions task is to create new files or copy old ones when a new
version is requested. It is called from mknod, truncate and write. The
result is always to ensure the symlink to the file exists and may be
to either create an entirely new file directory along with a new
version and lock file, create just a new version or do nothing at all.

The new directory is obviously created if nothing exist already and
will only be the case on mknod calls. A new version is created on most
truncates and the correct version number for the new version is given
by looking at the existing versions in the directory and incrementing
this number. Since we pr design skip a number if a file has been
deleted, a check is made if the symlink to the newest version exists.

It might also choose to not create a new version at all, but use the
oldest in the case where the newest versions filesize is 0. The
rationale behind this is an empty version is not interesting to have
at all.

Lastly a the symlink is updated to point to the new version if needed.

\subsection{Cleaner}
Write me!

\subsection{Mapping logical to physical files}

A core part of our implementation is mapping a logical file path.
There are three interestingly distinct kinds of file paths:

\begin{description}
\item[Plain path,] for example \texttt{/foo/bar/baz} which is a
  reference to the file \texttt{baz} in the folder \texttt{/foo/bar}.
\item[A wildcard directory reference,] such as \texttt{/foo/bar;*},
  which is a reference to directory \texttt{bar} in the folder
  \texttt{/foo}, and showing all versions of files in the directory.
\item[A specificion revision reference,] such as
  \texttt{/foo/bar/baz;0}, which would be version $0$ of the file
  \texttt{baz} in the folder \texttt{/foo/bar}.
\end{description}

All can be be divided into \textit{parts}, separated by slashes, and
the mapping operates by processing each path in turn.  As all logical
files and folders correspond to a symbolic link by their logical name,
we can merely read the destination (with the \texttt{readlink()}
system call) of each link in turn.  If the part contains a version
reference, the symlink (which points to the newest version) is not
naively followed, but the desired version is extracted from the
corresponding directory.  If the part is a wildcard directory
reference, everything past the semicolon is ignored for finding the
destination: the decision to show all versions of files is not
important to the path mapping itself, as they are all stored in the
same underlying directory anyway, and is instead handled in the
implementation of the \texttt{readdir} operation. If a symlink by the
name cannot be found, it means that the path as a whole refers to a
destination that does not exist, and a \textit{file not found} error
is signalled.

The implementation of this is contained in the \texttt{convert\_path}
method.

\end{document}
