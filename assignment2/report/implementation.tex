\documentclass[12pt]{article}
 
\author{
        Troels Henriksen, Troels Visti Thrane, Kasper Middelboe Petersen
}
\date{\today}

\usepackage{graphicx}
 
\title{Implementation}
 
\begin{document}
 
\maketitle

\section{Implementation Summary}

We have implemented our design as a program we have termed
\textit{Mnemosyne}, after the Greek goddess of memory.

Mnemosyne works by exposing a file system at a specified \textit{mount
  point}, while storing (and reading) data from a given \textit{root
  directory}.  This latter directory must be empty at the onset, and
only Mnemosyne should ever be allowed to modify its contents, as our
system is not robust in the face of data corruption.  Ideally, the
directory would be hidden away in some system folder (somewhere in
\texttt{/var} would be suitable on Unix systems), where there is no
risk the user might accidentally destroy it.

We make use of a Python wrapper around FUSE (of which there are
several, but we use the one that is in Debian under the name
\texttt{python-fuse}).  The general structure is that of defining a
subclass of \texttt{Fuse} that defines methods corresponding to FUSE
operations (\texttt{readdir}, \texttt{open}, \texttt{rename}, etc).
Additionally, we also wrote an entirely separate program for cleaning
the revision history of old entries (implemented in
\texttt{cleaner.py}).  This is simpler than running cleanup threads in
the main filesystem process, and our file-based locking mechanism
ensures that we do not get race conditions.

Mnemosyne is started as follows:\\
\texttt{python mnemosyne.py \textit{mount-point} -o
  root=\textit{root-directory}}.

\section{General Summary}

We have implemented every part of our design, with no known deviations
or bugs.  We have not rigorously tested our implementation, however,
instead relying on relatively simplistic empirical methods (``try it
and see if it breaks''), so we will hesitate to claim perfection in
all areas of our work.

\section{Interesting aspects}

\subsection{add\_version}

This functions task is to create new files or copy old ones when a new
version is requested. It is called from mknod, truncate and write. The
result is always to ensure the symlink to the file exists and may be
to either create an entirely new file directory along with a new
version and lock file, create just a new version or do nothing at all.

The new directory is obviously created if nothing exist already and
will only be the case on mknod calls. A new version is created on most
truncates and the correct version number for the new version is given
by looking at the existing versions in the directory and incrementing
this number. Since we pr design skip a number if a file has been
deleted, a check is made if the symlink to the newest version exists.

It might also choose to not create a new version at all, but use the
oldest in the case where the newest versions filesize is 0. The
rationale behind this is an empty version is not interesting to have
at all.

Lastly a the symlink is updated to point to the new version if needed.

\subsection{Cleaner}

A version controlled filesystem that keeps a copy of all revision is bound to
run out of space at some point. Even if we had used delta-compression to
minimize the size of each revision we would only be extending the time we had
before running out of space.

To try to avoid this we have decided to implement a process that searches
through all the files in the filesystem and deletes revisions that are unlikely
to be needed.

When the class is created it start at the root directory and lists all files
anddirectories. For each file in the directory it examines the revisions of the
file.
This is done by getting a list of the revisions and looping through them
starting with the oldest first.
The following rules are used to identify revisions unlikely to be needed:

\begin{itemize}
\item It is older than one month. Newer than one month and they are more likely
to be needed.
\item It has a newer revision that is less than one hour old
\end{itemize}

Both of these are configurable as mentioned in the design-report. The
configuration is done via the arguments given when calling the cleaner. For
simplicity we have decided that the user should setup a cronjob to start the
job, so when setting up the cronjob the two variables above can be controlled.


\subsection{Mapping logical to physical files}

A core part of our implementation is mapping a logical file path.
There are three interestingly distinct kinds of file paths:

\begin{description}
\item[Plain path,] for example \texttt{/foo/bar/baz} which is a
  reference to the file \texttt{baz} in the folder \texttt{/foo/bar}.
\item[A wildcard directory reference,] such as \texttt{/foo/bar;*},
  which is a reference to directory \texttt{bar} in the folder
  \texttt{/foo}, and showing all versions of files in the directory.
\item[A specificion revision reference,] such as
  \texttt{/foo/bar/baz;0}, which would be version $0$ of the file
  \texttt{baz} in the folder \texttt{/foo/bar}.
\end{description}

All can be be divided into \textit{parts}, separated by slashes, and
the mapping operates by processing each path in turn.  As all logical
files and folders correspond to a symbolic link by their logical name,
we can merely read the destination (with the \texttt{readlink()}
system call) of each link in turn.  If the part contains a version
reference, the symlink (which points to the newest version) is not
naively followed, but the desired version is extracted from the
corresponding directory.  If the part is a wildcard directory
reference, everything past the semicolon is ignored for finding the
destination: the decision to show all versions of files is not
important to the path mapping itself, as they are all stored in the
same underlying directory anyway, and is instead handled in the
implementation of the \texttt{readdir} operation. If a symlink by the
name cannot be found, it means that the path as a whole refers to a
destination that does not exist, and a \textit{file not found} error
is signalled.

The implementation of this is contained in the \texttt{convert\_path}
method.

\end{document}
